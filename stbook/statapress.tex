% statapress.tex
% version 1.1.3  14dec2005%
% Written by:  Jeff Pitblado
% This is a user's guide to statapress.cls, the Stata Press document class.

\section{The indexes}

\texttt{statapress.cls} automatically defines two types of indexes:  author
and subject.  Any string of words may be added to either index using the
\verb+\index+ macro.  For example, we use \verb+\index{subject}{indexes}+ to
add ``indexes''
%
\index{subject}{indexes}
%
to the subject index.  Next we use%
%
\verb+\index{author}{Knuth, D.~E.}+
%
to add ``Knuth, D.~E.''
%
\index{author}{Knuth, D.~E.}
%
to the author index as we cite \cite{texbook}.

The \verb+\stbkAuthorIndex+ and \verb+\stbkSubjectIndex+ macros generate
the section containing each respective index.  The \textsf{makeindex}
command is required; it reformats the raw index data into a \texttt{.ind}
sorted index data file.

\endinput
