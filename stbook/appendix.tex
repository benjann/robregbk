\appendix
\part{Appendix}
\label{part:appendix}

\chapter{Syntax and options}

\section{Robust statistics (robstat)}
\label{sec:syntax:robstat}                                                      \todo{This section will be replaced.
                                                                                The plan is to collect the different
                                                                                tools in new command \stcmd{robstat}}

\subsection{Classical estimators}

Classical estimators are readily available in Stata via the \texttt{summarize}
command using the \texttt{detail} option.

\subsection{Quantile-based estimators}

Quantile based estimators are not readily available in Stata but can easily be
calculated using the \texttt{centile} command. For example, for a statistical
series $x$, we compute:

\begin{description}
\item the \emph{median} $Q_{0.5;n}$ by
\end{description}

\texttt{centile x, centile(50)}

\texttt{local med=r(c\_1)}

\begin{description}
\item the \emph{corrected interquartile range} $\stsc{IQR}_c$ by
\end{description}

\texttt{centile x, centile(25 75)}

\texttt{local iqr=(r(c\_2)-r(c\_1))*0.7413}

\begin{description}
\item the Yule and Kendall skewness coefficient $\stsc{SK}_{0.25;n}$ by
\end{description}

\texttt{centile x, centile(25 50 75)}

\texttt{local sk=(r(c\_1)+r(c\_3)-2*r(c\_2))/(r(c\_2)-r(c\_1))}

\begin{description}
\item the quantile tail weight measures $\stsc{LQW}_{0.25;n}$ and
$\stsc{RQW}_{0.25;n}$ by
\end{description}

\texttt{centile body, centile(12.5 25 37.5 62.5 75 87.5)}

\texttt{local lqw=-(r(c\_1)+r(c\_3)-2*r(c\_2))/(r(c\_3)-r(c\_1))}

\texttt{local rqw=(r(c\_4)+r(c\_6)-2*r(c\_5))/(r(c\_6)-r(c\_4))}

\subsection{Pairwise-based estimators}

As explained in the previous sections, the location, scale, skewness and tails
heaviness estimators based on pairwise combinations or comparisons of the
observations are particularly interesting because they perform better than the
classical or quantile-based estimators, namely in terms of robustness. But
they require at first sight a heavy computation time of an order of $n^2$.
Fortunately, as explained in \cite{Croux.Rousseeuw (1992)}, an efficient
algorithm proposed by Johnson and Mizoguchi (see \cite{Johnson.Mizoguchi
(1978)}) allows to substantially reduce the computation time to an order of
$n\log n$ and, hence, allows to use these robust estimators even in very large
datasets. This algorithm has been programmed in Stata (cf.\
\cite{Gelade.Verardi.Vermandele (2013)}) and is involved in the Stata
computation of the pairwise-based estimators.

For these pairwise-based estimators, the Stata commands are \texttt{hl} (for
the Hodges-Lehman location estimator $\stsc{HL}_n$), \texttt{qn} (for the
Rousseeuw and Croux $Q_n$ estimator), \texttt{medcouple} (for Brys'
medcouple $\stsc{MC}_n$, left medcouple $\stsc{LMC}_n$ and right
medcouple $\stsc{RMC}_n$).\ The corresponding syntaxes are:

\underline{\textbf{Location}}

\medskip

\textbf{Title}

hl -- Hodges and Lehman (1963) robust measure of location

\smallskip

\textbf{Syntax}

hl varname [if] [in]

\bigskip

\underline{\textbf{Dispersion}}

\medskip

\textbf{Title}

qn -- Rousseeuw and Croux (1993) robust measure of dispersion

\smallskip

\textbf{Syntax}

qn varname [if] [in]

\bigskip

\underline{\textbf{Skewness and heavyness of the tails}}

\medskip

\textbf{Title}

medcouple -- medcouple measure of asymmetry and heaviness of the tails

\smallskip

\textbf{Syntax}

medcouple varname [if] [in] [, lmc rmc nomc]

\bigskip

----------------------------------------------------------------------------------

\textbf{options description}

----------------------------------------------------------------------------------

\underline{Main} \smallskip

\texttt{lmc} specifies to calculate the medcouple only for those informations
smaller than the median. This is an indicator of the heaviness of the left tail.

\texttt{rmc} Specifies to calculate the medcouple only for those informations
larger than the median. This is an indicator of the heaviness of the right tail.

\texttt{nomc} Specifies not to calculate the global medcouple. This is for
instance useful when one is only interested in the heaviness of the tails.

\subsection{Normality tests}

The Stata command \textquotedblleft\texttt{robjb} \textit{varname} [if]
[in]\textquotedblright\ has been created to apply these robust tests of
normality. This command implements the test considering both skewness and
heaviness of tails by default. Two mutually exclusive options are available:
\texttt{skewness} and \texttt{kurtosis}. If the former is used, a test based
exclusively on the skewness is performed while if the latter is called, a test
based exclusively on the heaviness of the tails is performed.

\subsection{Boxplot}

DESCRIBE\ HERE\ THE\ COMMAND, FOR\ THE\ MOMENT\ WE\ HAVE

\bigskip

\textbf{Title}

box\_out - Boxplot for skewed and/or heavy-tailed distributions

\textbf{Syntax}

box\_out varname [if] [in] [, out(varname) bdp(\#) perc(\#) nograph]

\underline{\textbf{Description}}

box\_out Creates the boxplot for skewed and/or heavy-tailed distributions

----------------------------------------------------------------------------------

\textbf{options description}

----------------------------------------------------------------------------------

\underline{Main} \smallskip

\texttt{out(varname)} Identifies the new variable to be created to identify
individuals outside the fence defined by the whiskers

\texttt{bdp(integer)} Sets the desired Break-down point (in \%). It is 10\% by default

\texttt{perc(real)} Sets the desired percentage of points outside the whiskers
in case of uncontaminated data.

\texttt{nograph} Suppresses the graph

box\_out saves the following in e():

\texttt{e(g) }Estimated skewness parameter of the underlying Tukey g and h distribution

\texttt{e(h)} Estimated elongation parameter of the underlying Tukey g and h distribution

\texttt{e(upperW)} Value of the upper whisker

\texttt{e(lowerW)} Value of the lower whisker




\section{Robust linear regression (robreg)}
\label{sec:syntax:robreg}

\stcmd{robreg} provides a number of robust estimators for linear
regression models. The syntax of \stcmd{robreg} is:

\begin{stsyntax}
    robreg 
    \stit{estimator} 
    \depvar\
    \optindepvars\
    \optif\
    \optin\
    %\optweight\
    \optional{, \stit{options}}
\end{stsyntax}

\noindent
where \stit{estimator} is one of the following:

\hangpara
    \stcmd{mm} to fit the efficient high breakdown \stsc{MM} estimator proposed
    by \citet{yohai:1987}. On the first stage, a high breakdown \stsc{S}
    estimator is applied to estimate the residual scale and derive starting
    values for the coefficients vector. On the second stage, an efficient
    bisquare \stsc{M} estimator is applied to obtain the final coefficient
    estimates.

\hangpara
    \stcmd{gm} \alert{to be implemented}

\hangpara
    \stcmd{m} to fit regression \stcmd{M} estimators \citep{huber73} using
    iteratively reweighted least squares (\stsc{IRWLS}).

\hangpara
    \stcmd{s} to fit the high breakdown \stcmd{S} estimator introduced by
    \citet{rousseeuw:yohai:1984} using the fast algorithm by
    \cite{salibian:yohai:2006} with a nonsingular subsampling refinement as
    proposed by \citet{Koller:2012}.

\hangpara
    \stcmd{lms}, \stcmd{lqs}, or \stcmd{lts} to fit the least median of squares
    (\stsc{LMS}), the least quantile of squares (\stsc{LQS}; a generalization
    of \stsc{LMS}), or the least trimmed squares (\stsc{LTS}) estimator,
    respectively \citep{rousseeuw:leroy:1987}. Estimation is carried out using
    simple resampling without local improvement (e.g.\
    \citealp[197]{rousseeuw:leroy:1987}). Computation of standard errors is not
    supported for \stcmd{lms}, \stcmd{lqs}, and \stcmd{lts}.

\noindent \stcmd{robreg} without arguments replays the previous results; see
\uref{20.3 Replaying prior results}. \stcmd{robreg} saves its results in
\stcmd{e()}, so that post estimation commands can be applied; see \uref{20
Estimation and postestimation commands}. Type \stcmd{ereturn list} to list the
\stcmd{e()}-returns after estimation; see \pref{ereturn}.

\subsection{Options for robreg mm}

\subsubsection{Main}

\hangpara
    \stcmd{\underbar{eff}iciency(\num)} sets the gaussian efficiency of the \stsc{MM}
    estimator (i.e., the asymptotic relative efficiency compared to the
    \stsc{LS} or \stsc{ML} estimator in case of i.i.d.\ normal errors). The
    efficiency is determined by appropriate choice of the tuning constant for
    the bisquare \stsc{M} estimator in the second stage of the \stsc{MM}
    algorithm. \num\ must be between 0.1 and 99.9. The default for the
    \stsc{MM} estimator is \stcmd{efficiency(85)}, as suggested by
    \citet[144]{maronna:etal:2006}.

\hangpara
    \stcmd{bp(\num)} sets the breakdown point of the \stsc{MM} estimator. The
    breakdown point is determined by appropriate choice of the tuning constant
    for the \stsc{S} estimator in the first stage of the \stsc{MM} algorithm.
    \num\ must be between 1 and 50. The default is \stcmd{bp(50)}.

\hangpara
    \stcmd{\underbar{haus}man} performs a generalized Hausman test of the \stsc{MM}
    estimate against the \stsc{S} estimate. A significant Hausman test
    indicates that the \stsc{MM} estimate significantly deviates from the
    \stsc{S} estimate.


\subsubsection{Biweight M estimate}

\hangpara
    \stcmd{k(\num)} specifies the tuning constant for the bisquare \stsc{M}
    estimator in the second stage of the \stsc{MM} algorithm. \stcmd{k()} not
    allowed if \stcmd{efficiency()} is specified.

\hangpara
    \stcmd{\underbar{tol}erance(\num)} specifies the tolerance for the weights of the
    \stsc{IRWLS} algorithm used to fit the bisquare \stsc{M} estimator. When
    the maximum absolute change in the weights from one iteration to the next
    is less than or equal to \stcmd{tolerance()}, the convergence criterion is
    satisfied. The default is \stcmd{tolerance(1e-6)}.

\hangpara
    \stcmd{\underbar{iter}ate(\num)} specifies the maximum number of iterations for the
    IRWLS algorithm used to fit the bisquare \stsc{M} estimator. If convergence
    is not reached within \stcmd{iterate()} iterations, the algorithm stops and
    returns error. The default is \stcmd{iterate(16000)} or as set by
    \stcmd{set maxiter} (see \rref{maximize}).

\hangpara
    \stcmd{relax} causes the \stsc{IRWLS} algorithm to return the current
    results instead of returning error if convergence is not reached.

\hangpara
    \stcmd{\dunderbar{g}enerate(\stit{newvar})} stores the final weights of the
    \stsc{IRWLS} algorithm in variable \stit{newvar}.

\hangpara
    \stcmd{\underbar{re}place} permits \stcmd{robreg} to overwrite existing variables.

\subsubsection{Initial S estimate}

\hangpara
    \stcmd{\underbar{n}samp(\num)} specifies the number of trial samples for the search
    algorithm of the \stsc{S} estimator in the first stage of the \stsc{MM}
    algorithm. The default value is determined according to formula
    $
    \lceil \ln(\alpha) / \ln(1 - (1 - \varepsilon)^p) \rceil
    $
    within a range of 50 to 10000, where $p$ is the number of coefficients in
    the model and $\alpha = 0.01$ and $\varepsilon = 0.2$ (see
    \citealp{salibian:yohai:2006} for a justification of the formula). The
    default values for $\alpha$ and $\varepsilon$ can be changed via
    \stcmd{sopts()} (see below).

\hangpara
    \stcmd{\underbar{s}opts(\stit{options})} specifies additional options to be passed
    through to the \stsc{S} estimator. See the section on options for
    \stcmd{robreg s} below.

\hangpara
    \stcmd{save(\stit{name})} saves the results of the \stsc{S} estimator under
    \stit{name} using \stcmd{estimates store} (see \rref{estimates}).

\subsubsection{Standard errors}

\hangpara
    \stcmd{vce(\underbar{nor}obust)} causes standard errors to be computed
    using traditional formulas assuming constant error variance. The default
    is to compute robust standard errors as suggested by \citet{Croux:2003}
    (using formula $\stsc{Avar}_1$; the traditional formula is equivalent to
    $\stsc{Avar}_{2s}$).

\hangpara
    \stcmd{\underbar{nor}obust} is a synonym for \stcmd{vce(norobust)}

\subsubsection{Reporting}

\hangpara
    \stcmd{\underbar{l}evel(\num)} specifies the level for confidence intervals. The
    default is \stcmd{level(95)} or as set by \stcmd{set level} (see
    \rref{level}).

\hangpara
    \stcmd{first} causes the first stage \stsc{S} estimate to be displayed.

\hangpara
    \stcmd{\underbar{nodot}s} suppresses the progress dots of the \stsc{S} estimator
    search algorithm.

\hangpara
    \stcmd{\underbar{lo}g} displays the iteration log of the second stage \stsc{IRWLS}
    algorithm.
    
\hangpara
    \stit{display\_options} are various display options; see \rref{estimation
    options}.

\subsection{Options for robreg gm}

\alert{To be completed.}

\subsection{Options for robreg m}

\subsubsection{Main}

\hangpara
    \stcmd{\underbar{h}uber} causes the Huber objective function to be used
    (monotone \stsc{M} estimator). This is the default.

\hangpara
    \stcmd{\underbar{bi}weight} causes the biweight or bisquare objective function to be
    used (redescending \stsc{M} estimator). \stcmd{bisquare} is a synonym for
    \stcmd{biweight}. The solution of a redescending \stsc{M} estimator may
    depend on the starting values.

\hangpara
    \stcmd{\underbar{eff}iciency(\num)} sets the gaussian efficiency (i.e.\ the asymptotic
    relative efficiency compared to the \stsc{LS} or \stsc{ML} estimator in
    case of i.i.d.\ normal errors) by appropriate choice of the tuning
    constant. \num\ must be between 63.7 and 99.9 for \stcmd{huber} and between
    0.1 and 99.9 for \stcmd{biweight}. The default is \stcmd{efficiency(95)}.

\hangpara
    \stcmd{k(\num)} specifies the tuning constant. \stcmd{k()} not allowed if
    \stcmd{efficiency()} is specified.

\subsubsection{IRWLS algorithm}

\hangpara
    \stcmd{\underbar{tol}erance(\num)} specifies the tolerance for the weights of the
    \stsc{IRWLS} algorithm. When the maximum absolute change in the weights
    from one iteration to the next is less than or equal to
    \stcmd{tolerance()}, the convergence criterion is satisfied. The default is
    \stcmd{tolerance(1e-6)}.

\hangpara
    \stcmd{\underbar{iter}ate(\num)} specifies the maximum number of iterations for the
    \stsc{IRWLS} algorithm. If convergence is not reached within
    \stcmd{iterate()} iterations, the algorithm stops and returns error. The
    default is \stcmd{iterate(16000)} or as set by \stcmd{set maxiter} (see
    \rref{maximize}).

\hangpara
    \stcmd{relax} causes the \stsc{IRWLS} algorithm to return the current results
    instead of returning error if convergence is not reached. For example,
    to fit a one-step \stsc{M} estimate specify \stcmd{relax} together with
    \stcmd{iterate(1)}.

\hangpara
    \stcmd{\dunderbar{g}enerate(\stit{newvar})} stores the final weights of the
    \stsc{IRWLS} algorithm in variable \stit{newvar}.

\hangpara
    \stcmd{\underbar{re}place} permits \stcmd{robreg} to overwrite existing variables.

\subsubsection{Initial estimate}

\hangpara
    \stcmd{init(\textit{arg})} determines the choice of the initial estimate
    that provides the starting values for the \stsc{IRWLS} algorithm.
    \stit{arg} may be \stcmd{lav} for the \stsc{LAV} estimator (a.k.a.\ median
    regression; fitted using \stcmd{qreg}, see \rref{qreg}), \stcmd{ols} for
    the least squares estimator (fitted using \stcmd{regress}, see
    \rref{regress}), \stit{name} for an estimation set stored under
    \stit{name}, or \stcmd{.} for the currently active estimation results. The
    default is \stcmd{init(lav)}.

\hangpara
    \stcmd{save(\stit{name})} saves initial \stcmd{lav} or \stcmd{ols} estimate
    under \stit{name} using \stcmd{estimates store} (see \rref{estimates}).

\subsubsection{Scale estimate}

\hangpara
    \stcmd{\underbar{s}cale(\num)} provides a preliminary value for the residual scale
    that will be held constant. The default is to use the normalized median of
    the $(n-p)$ largest absolute residuals from the initial fit, where $n$ is
    the sample size and $p$ is the number of coefficients, as an estimate of
    the residual scale (\stsc{MADN}).

\hangpara
    \stcmd{\dunderbar{update}scale} causes the \stsc{MADN} scale estimate to be updated in
    each iteration of the \stsc{IRWLS} algorithm. \stcmd{updatescale} has no
    effect if \stcmd{scale()} is specified.

\hangpara
    \stcmd{\underbar{cen}ter} causes the \stsc{MADN} scale estimate to be computed based
    on median centered residuals. \stcmd{center} has no effect if
    \stcmd{scale()} is specified.

\subsubsection{Standard errors}

\hangpara
    \stcmd{vce(\underbar{nor}obust)} causes standard errors to be computed
    using traditional formulas assuming constant error variance. The default
    is to compute robust standard errors as suggested by \citet{Croux:2003}
    (using formula $\stsc{Avar}_1$; the traditional formula is equivalent to
    $\stsc{Avar}_{2s}$).

\hangpara
    \stcmd{vce(pv)} causes traditional standard errors to be computed using the
    pseudo-values approach \citep{street.etal.AmStat.1988}. \stcmd{vce(pv)} is
    equivalent to \stcmd{vce(norobust)} but includes some small sample
    correction.

\hangpara
    \stcmd{\underbar{nor}obust} is a synonym for \stcmd{vce(norobust)}

\hangpara
    \stcmd{nose} skips the computation of standard errors.

\subsubsection{Reporting}

\hangpara
    \stcmd{\underbar{l}evel(\num)} specifies the level for confidence intervals. The
    default is \stcmd{level(95)} or as set by \stcmd{set level} (see
    \rref{level}).

\hangpara
    \stcmd{first} causes the initial estimate to be displayed.

\hangpara
    \stcmd{\underbar{lo}g} displays the iteration log of the second stage \stsc{IRWLS}
    algorithm.

\hangpara
    \stit{display\_options} are various display options; see \rref{estimation
    options}.

\subsection{Options for robreg s}

\subsubsection{Main}

\hangpara
    \stcmd{bp(\num)} sets the breakdown point by appropriate choice of the
    tuning constant (this also determines the gaussian efficiency). \num\
    must be between 1 and 50. The default is \stcmd{bp(50)}.

\hangpara
    \stcmd{k(\num)} specifies the tuning constant. \stcmd{k()} not allowed if
    \stcmd{bp()} is specified.
    
\hangpara
    \stcmd{\underbar{haus}man} performs a generalized Hausman test of the least squares
    estimate against the \stsc{S} estimate. A significant Hausman test
    indicates that the least squares estimate significantly deviates from the
    \stsc{S} estimate.

\subsubsection{Resampling algorithm}

\hangpara
    \stcmd{\underbar{c}ategorical(\stit{varlist})} specifies the variables to be treated
    as categorical. The default is to detect and treat all dummy variables as
    categorical. If \stcmd{categorical()} is specified, the detection of dummy
    variables is deactivated and only the variables identified by
    \stcmd{categorical()} are treated as categorical. Variables from
    \stcmd{categorical()} that are not found in \stit{indepvars} will be
    automatically added to the end of \stit{indepvars}. \stcmd{categorical()}
    may contain factor variables; see \uref{11.4.3 Factor variables}.
    \alert{I changed the default behavior, if I remember right. Needs to be updated.}

\hangpara
    \stcmd{\underbar{noc}ategorical} treats all variables as continuous.

\hangpara
    \stcmd{\dunderbar{nonsing}ular} use nonsingular subsampling \alert{needs to be completed; 
    nonsingular should be default}

\hangpara
    \stcmd{\underbar{n}samp(\num)} specifies the number of trial samples for the search
    algorithm. The default value is determined according to formula
    $
    \lceil \ln(\alpha) / \ln(1 - (1 - \varepsilon)^p) \rceil
    $
    within a range of 50 to 10000, where $p$ is the number of coefficients in
    the model and $\alpha$ and $\varepsilon$ are set by \stcmd{alpha()} and
    \stcmd{epsilon()} (see \citealp{salibian:yohai:2006} for a justification of
    the formula).

\hangpara
    \stcmd{alpha(\num)} specifies the maximum admissible risk of drawing a set
    of samples of which none is free of outliers. This is a parameter in the
    formula for the computation of the required number samples (see above). The
    default is \stcmd{alpha(0.01)} (i.e.\ 1 percent). \stcmd{alpha()} has no
    effect if \stcmd{nsamp()} is specified.

\hangpara
    \stcmd{\dunderbar{eps}ilon(\num)} specifies the assumed maximum fraction of
    contaminated data. This is a parameter in the formula for the computation
    of the required number samples (see above). The default is
    \stcmd{epsilon(0.2)} (i.e.\ 20 percent). \stcmd{epsilon()} has no effect if
    \stcmd{nsamp()} is specified.

\hangpara
    \stcmd{\underbar{nk}eep(\num)} specifies the number of best candidates to be
    kept for final refinement. The default is \stcmd{nkeep(2)}.

\hangpara
    \stcmd{\dunderbar{rstep}s(\num)} specifies the number of local improvement steps
    applied to the candidates. The default is \stcmd{rsteps(1)}.

\hangpara
    \stcmd{\underbar{stol}erance(\num)} specifies the tolerance for the scale estimate of
    the candidates. When the absolute relative change in the scale from one
    iteration to the next is less than or equal to \stcmd{stolerance()}, the
    convergence criterion is satisfied. The default is \stcmd{stolerance(1e-6)}.

\hangpara
    \stcmd{\underbar{siter}ate(\num)} specifies the maximum number of iterations for the
    scale estimate of the candidates. If convergence is not reached within
    \stcmd{siterate()} iterations, the algorithm stops and returns error. The
    default is \stcmd{siterate(16000)} or as set by \stcmd{set maxiter} (see
    \rref{maximize}).

\hangpara
    \stcmd{\underbar{tol}erance(\num)} specifies the tolerance for the coefficients in the
    refinement \stsc{IRWLS} algorithm. When the maximum relative change in the
    coefficient vector from one iteration to the next is less than or equal to
    \stcmd{tolerance()}, the convergence criterion is satisfied. The default is
    \stcmd{tolerance(1e-6)}.

\hangpara
    \stcmd{\underbar{iter}ate(\num)} specifies the maximum number of iterations for the
    refinement \stsc{IRWLS} algorithm. If convergence is not reached within
    \stcmd{iterate()} iterations, the algorithm stops and returns error. The
    default is \stcmd{iterate(16000)} or as set by \stcmd{set maxiter} (see
    \rref{maximize}).

\hangpara
    \stcmd{\dunderbar{sstep}s(\num)} specifies the number of approximation steps for the
    scale estimate within each \stsc{RWLS} iteration. The default is
    \stcmd{ssteps(1)}.

\hangpara
    \stcmd{\dunderbar{srstep}s(\num)} n of iterations for p-subset scale; default: until
    convergence \alert{revise!}

\hangpara
    \stcmd{\underbar{ceff}iciency(\num)} efficiency of M for catvars; default
    \stcmd{cefficiency(95)} \alert{revise!}

\hangpara
    \stcmd{ck(\num)} k of M for catvars; not allowed with \stcmd{cefficiency()}
    \alert{revise!}

\hangpara
    \stcmd{\dunderbar{cstep}s(\num)} approx. steps for catvar M within p-subset; default
    \stcmd{cstep(0)} \alert{revise!}

\hangpara
    \stcmd{\underbar{noxr}esid} do not residualize continuous variables \alert{revise!}

\hangpara
    \stcmd{\dunderbar{fstep}s(\num)} approx. iteration in final backfit rounds; default:
    until convergence \alert{revise!}

\hangpara
    \stcmd{\underbar{nback}fit(\num)} n of final backfit rounds; default: 20
    \alert{revise!}

\hangpara
    \stcmd{\underbar{nobr}eak} do not exit final backfitting if scale increases
    \alert{revise!}

\hangpara
    \stcmd{\dunderbar{g}enerate(\stit{newvar})} stores the final \stsc{IRWLS} weights from
    the best solution in variable \stit{newvar}.

\hangpara
    \stcmd{\underbar{re}place} permits \stcmd{robreg} to overwrite existing variables.

\subsubsection{Standard errors}

\hangpara
    \stcmd{vce(\underbar{nor}obust)} causes standard errors to be computed using
    traditional formulas assuming constant error variance. The default is to
    compute robust standard errors as suggested by \citet{Croux:2003} (using
    formula $\stsc{Avar}_1$; the traditional formula is equivalent to
    $\stsc{Avar}_{2s}$).
    
\hangpara
    \stcmd{\underbar{nor}obust} is a synonym for \stcmd{vce(norobust)}

\hangpara
    \stcmd{nose} skips the computation of standard errors.

\subsubsection{Reporting}

\hangpara
    \stcmd{\underbar{l}evel(\num)} specifies the level for confidence intervals. The
    default is \stcmd{level(95)} or as set by \stcmd{set level} (see
    \rref{level}).

\hangpara
    \stcmd{\underbar{nodot}s} suppresses the progress dots of the search algorithm.

\hangpara
    \stit{display\_options} are various display options; see \rref{estimation
    options}.

\subsection{Options for robreg lms, robreg lqs, and robreg lts}

\subsubsection{Main}

\hangpara
    \stcmd{bp(\num)} sets the breakdown point, where \num\ may be in $(0,0.5]$. \stcmd{bp()}
    determines the $h$ parameter for the \stsc{LQS} and \stsc{LTS} estimators as
    $
    h = \lfloor(1-\num) \cdot n\rfloor + \lfloor\num \cdot (p + 1)\rfloor
    $
    where $n$ is the sample size and $p$ is the number of coefficients. The
    default is \stcmd{bp(0.5)}. \stcmd{bp()} is not allowed with
    \stcmd{robreg lms}.

\subsubsection{Resampling algorithm}

\hangpara
    \stcmd{\underbar{n}samp(\num)} specifies the number of trial samples for the search
    algorithm. The default value is determined according to formula
    $
    \lceil \ln(\alpha) / \ln(1 - (1 - \varepsilon)^p) \rceil
    $
    within a range of 50 to 10000, where $p$ is the number of coefficients in
    the model and $\alpha$ and $\varepsilon$ are set by \stcmd{alpha()} and
    \stcmd{epsilon()}.

\hangpara
    \stcmd{alpha(\num)} specifies the maximum admissible risk of drawing a set
    of samples of which none is free of outliers. This is a parameter in the
    formula for the computation of the required number samples (see above). The
    default is \stcmd{alpha(0.01)} (i.e.\ 1 percent). \stcmd{alpha()} has no
    effect if \stcmd{nsamp()} is specified.

\hangpara
    \stcmd{\dunderbar{eps}ilon(\num)} specifies the assumed maximum fraction of
    contaminated data. This is a parameter in the formula for the computation
    of the required number samples (see above). The default is
    \stcmd{epsilon(0.2)} (i.e.\ 20 percent). \stcmd{epsilon()} has no effect if
    \stcmd{nsamp()} is specified.

\hangpara
    \stcmd{\dunderbar{g}enerate(\stit{newvar})} stores a variable \stit{newvar} that marks the
    minimizing trial sample.

\hangpara
    \stcmd{\underbar{re}place} permits \stcmd{robreg} to overwrite existing variables.

\subsubsection{Reporting}

\hangpara
    \stcmd{\underbar{nodot}s} suppresses the progress dots of the search algorithm.

\hangpara
    \stit{display\_options} are various display options; see \rref{estimation
    options}.

\section{Robust logistics regression (roblogit)}
\label{sec:syntax:roblogit}

\section{Robust multivariate statistics (robmv)}
\label{sec:syntax:robmv}